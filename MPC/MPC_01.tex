\documentclass[UTF8,a4paper,10pt]{ctexart}
\usepackage[left=2.50cm, right=2.50cm, top=2.50cm, bottom=2.50cm]{geometry} %页边距
\CTEXsetup[format={\Large\bfseries}]{section} %设置章标题居左
 
 
%%%%%%%%%%%%%%%%%%%%%%%
% -- text font --
% compile using Xelatex
%%%%%%%%%%%%%%%%%%%%%%%
% -- 中文字体 --
%\setmainfont{Microsoft YaHei}  % 微软雅黑
%\setmainfont{YouYuan}  % 幼圆    
%\setmainfont{NSimSun}  % 新宋体
%\setmainfont{KaiTi}    % 楷体
%\setmainfont{SimSun}   % 宋体
%\setmainfont{SimHei}   % 黑体
% -- 英文字体 --
%\usepackage{times}
%\usepackage{mathpazo}
%\usepackage{fourier}
%\usepackage{charter}
\usepackage{helvet}
 
 
\usepackage{amsmath, amsfonts, amssymb} % math equations, symbols
\usepackage[english]{babel}
\usepackage{color}      % color content
\usepackage{graphicx}   % import figures
\usepackage{url}        % hyperlinks
\usepackage{bm}         % bold type for equations
\usepackage{multirow}
\usepackage{booktabs}
\usepackage{epstopdf}
\usepackage{epsfig}
\usepackage{algorithm}
\usepackage{algorithmic}

\usepackage{easybmat}
\renewcommand{\algorithmicrequire}{ \textbf{Input:}}     % use Input in the format of Algorithm  
\renewcommand{\algorithmicensure}{ \textbf{Initialize:}} % use Initialize in the format of Algorithm  
\renewcommand{\algorithmicreturn}{ \textbf{Output:}}     % use Output in the format of Algorithm  
 
 
\usepackage{fancyhdr} %设置页眉、页脚
%\pagestyle{fancy}
\lhead{}
\chead{}
%\rhead{\includegraphics[width=1.2cm]{fig/ZJU_BLUE.eps}}
\lfoot{}
\cfoot{}
\rfoot{}
 
 
%%%%%%%%%%%%%%%%%%%%%%%
%  设置水印
%%%%%%%%%%%%%%%%%%%%%%%
%\usepackage{draftwatermark}         % 所有页加水印
%\usepackage[firstpage]{draftwatermark} % 只有第一页加水印
% \SetWatermarkText{Water-Mark}           % 设置水印内容
% \SetWatermarkText{\includegraphics{fig/ZJDX-WaterMark.eps}}         % 设置水印logo
% \SetWatermarkLightness{0.9}             % 设置水印透明度 0-1
% \SetWatermarkScale{1}                   % 设置水印大小 0-1    
 
\usepackage{hyperref} %bookmarks
\hypersetup{colorlinks, bookmarks, unicode} %unicode
 
 
 
\title{\textbf{MPC公式推导笔记}}
\author{ ColaForced}
\date{\today}
 
\begin{document}
    \maketitle
 
 
% \begin{abstract}
% 这是一篇中文小论文。这个部分用来写摘要。摘要的章标题默认是英文,还没找到改成中文的方法:(
% \end{abstract}


\subparagraph{1. 最优控制}

    为了系统在有约束条件下,系统某项表现达到最优,一般用代价函数最小表示。
    比如说,火箭升空燃料代价最小,如何设计控制输入。
    值得一提的是,选定目标不同,计算出的控制输入也不同,
    比如说,追求快速升空,则控制输入势必是往油门最大化。

    对于一个多目标相互约束问题,如汽车变道,
    一方面要求轨迹误差尽可能小,另一方面要求侧向控制输入尽可能小(为了舒适度)。
    在这种情况下,代价函数则是线性带权重的组合:
    \begin{equation}
        J = \int_{0}^{t}qe(s)^2 + ru(s)^2ds
    \end{equation}\label{model}
    其中,$u,e$分别表示控制输入和追踪误差的。$q,r$权重用于调整系统表现。

    对于一般线性系统,则可以表现为
        \begin{equation}
            J = \int_{0}^{t}E^TQE + U^TRUds
        \end{equation}
    对于这样的二次型最优化问题,应该是有相对成体系的计算方案。

\subparagraph{2. MPC算法}

    对于一个离散线性系统:
        \begin{equation}\label{model}
            X_{k+1} = AX_{k} + BU_{k}
        \end{equation}
    显然,若已知初始条件$X_{k}$,和控制输入$U_{k}$,则势必可以计算出$X_{k+1}$.

    
    计算思路:

        1. 观测/估计k时刻状态
        
        2. 最小化代价函数来设计输入, 代价函数是 
        \begin{equation}
            J = \sum_{k}^{N-1}({E_{K}^T Q E_{K} + U_{K}^T R U_{K}}) + E_N^TFE_N
        \end{equation}

        3. 仅用计算出的最优$U_k$作为输入,然后再次循环。
    
    Tips: 称之为滚动时域优化。

\subparagraph{3. MPC代价函数公式推导变形}

目标,滚动时域优化,当前难点在于解决最小化代价函数怎么计算。
既已知道LQR中二次型优化问题怎么计算, 
so, 当前思想就是想把MPC中的优化问题转化成二次型的形式。

先做一些声明:

\begin{equation}
    \hat{X}_k = [X_{(k|k)}, X_{(k+1|k)}, ..., X_{(k+N|k)}]^T
\end{equation}

\begin{equation}
    \hat{U}_k = [U_{(k|k)}, U_{(k+1|k)}, ..., U_{(k+N-1|k)}]^T
\end{equation}
其中$X_{(k|k)} = X_k$表示初始状态,
$ X_{(k+1|k)}$ 则是根据($\ref{model}$)和$U_{(k|k)},X_k$计算出来的,
相应地,$X_{(k+1|k)}, ..., X_{(k+N|k)}$是进一步迭代计算得到的。


误差函数定义:

    \begin{equation}\label{MPC_J}
        J = \sum_{k}^{N-1}{(X_{(K+i|k)}^T Q X_{(K+i|k)} + U_{(k+i|k)}^T R U_{(K+i|k)})} + X_{(k+N)}^T F X_{(k+N)}
    \end{equation}

为了带着目的去看推导,先给出最后推导出的代价函数形式:

    \begin{equation}\label{J_AIM}
        J = {\hat{X}_k^T G \hat{X}_k + \hat{U}_k^T H \hat{U}_k} + 2\hat{X}_k^T E \hat{U}_k
    \end{equation}
可以看到,就是为了转换成二次型的形式,把累加项转换成$\hat{X}_k, \hat{U}_k$整体项。

so,既然结果看到了,加上能用的信息显然就是动态模型(\ref{model})了,那么,就代公式吧!
  
    \begin{equation}
        \left\{
            \begin{array}{cc}
                X_{(k|k)} = X_{k},\\ \\
                X_{(k+1|k)} = AX_{k} + BU_{k|K},\\ \\
                X_{(k+2|k)} = A^2X_{k} + ABU_{k|k} + BU_{k+1|K},\\ \\
                \dots,\\ \\ 
                X_{(k+N|k)} = A^NX_{k} + A^{N-1}BU_{k|K} + \dots +  BU_{k+N-1|K}
            \end{array}
        \right.
    \end{equation}

把上式转换成

    \begin{equation}\label{MPC_TM}
        \hat{X}_k = M\hat{X}_k + C\hat{U}_k
    \end{equation}
其中,
\begin{equation}
    M=\left[
        \begin{BMAT}(rc){c}{ccccc}
            I  \\
            A  \\
            A^2  \\
            \vdots \\
            A^N
        \end{BMAT}
        \right],
\end{equation}

\begin{equation}
    C = \left[
        \begin{BMAT}(rc){cccc}{ccccc}
            0 & 0 & \dots & 0  \\
            B & 0 & \dots & 0 \\
            AB & B & \dots & 0 \\
            \vdots & \vdots & \vdots & B \\
            A^{N-1}B & A^{N-2}B & ... &...
        \end{BMAT}
        \right],
\end{equation}


在回顾公式(\ref{MPC_J}), 对其展开,再合并,可以得到

    \begin{equation}\label{MPC_J2}
        J = {\hat{X}_k^T \bar{Q}  \hat{X}_k + \hat{U}_k^T \bar{R} \hat{U}_k}
    \end{equation}
其中,
    \begin{equation}
        \bar{Q} = \left[
            \begin{BMAT}(rc){ccccc}{ccccc}
                Q & 0 & 0 &\dots & 0  \\
                0 & Q & 0 &\dots & 0 \\
                0 & 0 & Q & 0& 0 \\
                \vdots & \vdots & 0 &\vdots & 0 \\
                0 & 0 & 0 &0 &F
            \end{BMAT}
            \right],
    \end{equation}

    \begin{equation}
        \bar{R} = \left[
            \begin{BMAT}(rc){ccccc}{ccccc}
                R & 0 & 0 &\dots & 0  \\
                0 & R & 0 &\dots & 0 \\
                0 & 0 & R & 0& 0 \\
                \vdots & \vdots & 0 &\vdots & 0 \\
                0 & 0 & 0 &0 &R
            \end{BMAT}
            \right],
    \end{equation}


再把上式(\ref{MPC_J2})带入(\ref{MPC_TM})可以得到:

\begin{equation}
    J = \hat{X}_k^T M^T \bar{Q} M \hat{X}_K + \hat{X}_k^T M^T \bar{Q} C \hat{U}_k
    + \hat{U}_k^T C^T \bar{Q} M^T \hat{X}_k
    + \hat{U}_k^T C^T \bar{Q} C \hat{U}_k + \hat{U}_k^T \bar{R} \hat{U}_k
\end{equation}
再简化成(\ref{J_AIM})即可。


\subparagraph{4. 学习来源} https://www.bilibili.com/video/BV1SQ4y1Y7FG


    神仙UP主DR\_CAN, 我愿称之为YYDS!!!
    
\end{document}
